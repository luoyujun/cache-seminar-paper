\section{Abstract?}
% \section{Analysis of Algorithms}

% > However the efficiencies of any two "reasonable" implementations of a given algorithm
% > are related by a constant multiplicative factor called a *hidden constant*.
%      -- https://en.wikipedia.org/wiki/Analysis_of_algorithms

% Mathematical analysis of algorithms to obtain asymptotic upper bounds is nice.  Ignoring
% the non-uniform nature of memory hierarchies can yield pretty bad results, though.  What
% can be done about this?

We have seen that
\alts{
  {the hidden constant \alts{separating, distinguishing} the time complexities of two
  reasonable algorithms under asymptotic analysis can get quite big in the presence of a
  memory hierarchy.},
  {analyzing \alts{an algorithm's,} \alts{time, asymptotic, algorithmic} complexity,
  % This is not true: it doesn't matter for asymptotic complexity if there are different
  % types of memory.
  \x{under the usual \gls{ram} model, where a single uniform memory is \alts{presumed,
  assumed, supposed, premised},}
  can be quite inaccurate in the presence of a memory hierarchy.},
}
\alts{
  To \alts{escape, avert, prevent} having to rely \alts{only, entirely, exclusively} on
  empirical results, To overcome this problem,
},
abstract machine models taking the non-uniform \alts{memories of, nature of memory in}
real-world computers into account \alts{can be used, are used, have been proposed}.
% By whom?
One of these is the \emph{\gls{emm}}\x{\alts{
  ~\cite[5]{afmh},
  {, used \alts{in~\cite{afmh}, by \textcite{afmh}}.},
}}.

\subsection{External Memory Model}

% https://en.wikipedia.org/wiki/Random-access_machine

% TODO: go into more detail about the RAM model?
\alts[2]{
  {\alts{Establishing the basis of the, The} \gls{emm}, the widely used \gls{ram} model
  assumes \say{a \say{sufficiently} large uniform memory}~\cite[5]{afmh} with a constant
  access time.
  The},
  {The \gls{emm} is an extension of the \gls{ram} model.  While the latter assumes \say{a
  \say{sufficiently} large uniform memory}~\cite[5]{afmh} with a constant access time,
  the},
}
\gls{emm} \alts{divides, \alts{expands upon, extends} this by dividing} the memory into
\emph{internal} and \emph{external}.
\alts{The, Only the} internal memory is accessed directly, but its size is limited to
\(M\) \alts{items, words}.
The external memory is unbounded, but \alts{
  can only be accessed indirectly by loading data,
  data \x{is not accessed directly and} has to be loaded,
}
into internal memory \say{using \glslink{io}{I/Os} that move \(B\) contiguous
\alts{[items], words}}~\cite[5]{afmh}.

% > Although the word "I/O" suggests that external memory should be identified with disk
% > memory, we are free to choose any two levels of the memory hierarchy for internal and
% > external memory in the model.
%        -- Algorithms for Memory Hierarchies, page 6
The use of the term \emph{\acrshort{io}} here is somewhat non-standard.  While it suggests
\x{\alts{that, the}} external memory represents an \alts{
  {\gls{hdd} or \gls{ssd}},
  {\gls{hdd}, \gls{ssd} or similar},
}, it is
not constrained which \alts{physical storages, members of an actual memory hierarchy} are
\alts[2]{represented by, associated \alts{with, to}} \x{the} internal and external memory.
% Which members of an actual memory hierarchy are \alts{represented by, associated to} the
% internal and external memory is \alts{not constrained, as desired}.
If we choose the set of all CPU caches and the main memory, \(B\) becomes the cache line
size and \(M\) may be in the order of a few \si{\mebi\byte}.

The number of \acrshortpl{io} an algorithm requires in the \gls{emm} can augment the
information provided by standard asymptotic complexity analysis%
\x{, if \acrshort{io} is pivotal to an algorithm's runtime},
but \alts[2]{can't substitute, is no substitute for} measurements\x{~\cite[181]{afmh}}.
For example, the lower bound of \acrshortpl{io} needed for computing the sum of some
input of size \(N\), like in \cref{lst:ithare}, is
% See <http://erikdemaine.org/papers/BRICS2002/paper.pdf#page=7>.
\alts[3]{
  \(\lceil\nicefrac{N}{B}\rceil\),
  \(\left(\lfloor\nicefrac{N}{B}\rfloor + 2\right)\),
  \(\left(\lceil\nicefrac{N}{B}\rceil + 1\right)\),
}
in the \gls{emm}.\footnote{%
  TODO.
}
The linked list in that example probably takes almost \(N\) \acrshortpl{io},
though, since consecutive \alts{nodes, elements} are unlikely to fall into the same cache
line.  Thus, the predicted performance difference between \mintinline{cpp}{std::vector}
and \mintinline{cpp}{std::list} is at most \(B\), the number of items a cache line can
hold, which is \si{16} in this case.%
\footnote{%
  % TODO: elaborate on the size of `int`?
  A cache line is \alts{\si{64} bytes, \SI{64}{\byte}} on my laptop's CPU and an
  \mintinline{text}{int} \si{4} bytes with my compiler.
}
Recalling that the \alts{measured, actual} performance difference was
\num[round-mode=places, round-precision=0]{\speedup}, this is pretty inaccurate, but more
informative than saying both \x{algorithms'} data structure's time complexities for
traversal are \x{in} \(\Theta(N)\).

% An extensive repository of the theoretical lower bounds of \acrshortpl{io} needed in the
% \gls{emm} for many problems is available in the literature (for example \cite{afmh}).

Even with the simplifications made by the \gls{emm}, algorithmic analysis is usually only
done asymptotically: the number of \acrshortpl{io} is expressed \alts{in terms of, as}
\(\mathcal{O}\left(f\left(N, M, B\right)\right)\) or one of the related symbols.
Theoretical lower bounds of the \acrshortpl{io} needed in the \gls{emm} are available in
the literature (for example~\cite{afmh}) for many problems.

% This analysis only makes sense when \acrshort{io} is pivotal to an algorithm's
% efficiency.

% The objective is to \alts{
%   {design algorithms requiring a \alts{minimum number, minimum, minimal number} of
%   \acrshortpl{io}.},
%   {minimize the number of \alts{\acrshortpl{io}, I/Os} \alts{an algorithm needs,
%   needed}.},
% }

\x{
% Applying this model to \cref{lst:ithare} suggests
The \gls{emm} suggest that we can at best sum containers like in \cref{lst:ithare} using
\(\nicefrac{N}{B}\) \acrshortpl{io}, where \(N\) is the number of items to sum and \(B\)
the amount of them that fit in one block.
% Since there's 5000 containers of size 5000 and my cache line
% size is \SI{64}{B} and fits 8 \mintinline{cpp}{int} values:
% \begin{equation*}
%   N/B = 5000^2/8 = 3125000
% \end{equation*}
We can assume that the version with the list uses almost \(N\) \acrshortpl{io} because of
the random memory access.  Since a cache line fits 16 \mintinline{text}{int} values (\(B =
16\)) it
predicts a performance difference of 16.  This is not very accurate.
Analysis of \acrshortpl{io} needed by an algorithms \x{or data structure} may help inform
the decision of whether to consider it for empirical comparisons, though.
% that can be used prior to empirical comparisons.

% Apparently, the model is not practical for comparing these two data structures.
}

% Now for something completely different.
\x{
The \gls{emm} also suggests another approach to improve \cref{lst:ithare}, if we don't
want to \alts{trade, forgo, give up} the \x{asymptotically} constant time complexity of
insertions and deletions anywhere in the container: an \emph{unrolled list}.
}

% > In general it is best to hardcode cache line sizes at compile time by using the
% > `getconf` utility as in:
% >    gcc -DCLS=$(getconf LEVEL1_DCACHE_LINESIZE) ...
% > If the binaries are supposed to be generic, the largest cache line size should be
% > used.
%      -- Drepper, p. 50

% \subsubsection{Notes}
% \subsubsection{Caveats}
% \subsubsection{Limitations and Other Models}
\subsubsection{Limitations}

% Notes and potential problems:
% * The concept of I/Os that move B contiguous words directly translates to cache lines.
% * There's nothing equivalent to prefetching in the EMM.
% * > The main problem with hardware caches is that they use a fixed simplistic strategy
%   > for deciding which blocks are kept whereas the external memory model gives the
%   > programmer full control over the content of internal memory.
%        -- Algorithms for Memory Hierarchies, page 6
% * We lost associativity too.
% * And we don't model multiple levels of CPU caches, of course.

% Replacement policies.  Associativity.  More than two levels of memory.

% While the concept of \acrshortpl{io} directly models \alts{
%   most\footnote{%
%     TODO: not associativity though.
%   },
%   the
% }
% \alts{effects, existence} of cache lines,
% % are directly modeled by the concept of \acrshortpl{io},
% nothing in the \gls{emm} accounts for \x{those of} prefetching.

While the concept of \acrshortpl{io} directly models cache lines, most other
characteristics of memory hierarchies are ignored by the \gls{emm}.  For example:
\begin{itemize} % TODO: probably don't use a list structure.
  \item prefetching, or more generally the advantages of sequential access patterns,
  \item multi-level caches,
  \item the lack of direct control over the contents of caches, % (\alts{CPUs,
    % \glspl{cpu}} employ fixed, hardware-controlled replacement strategies)
  \item associativity,\footnote{%
      Associativity is not discussed in this paper; see \textcite{drepper2007} instead.
    }
  \item \gls{tlb}.
  % critical word stuff...
\end{itemize}
%
% > In the EMM it is generally assumed that the cost of I/Os is much greater than the cost
% > of computation, hence the design of the algorithm is usually motivated by the need to
% > minimise the number of I/Os.  However in the CMM and the IMM the relative miss
% > penalties are far smaller and we have to consider computation costs.
%      -- Algorithms for Memory Hierarchies, Section 8.6.4, page 188
%
% > The relative miss penalty is much lower for caches or the TLB than for disks, so
% > constant factors are important, we find that asymptotic analysis is not enough to
% > determine the performance of algorithms.
%      -- Algorithms for Memory Hierarchies, page 181
%
\alts{More fundamentally, Additionally}, the model's premise is that \acrshortpl{io} are
much more expensive than computation~\cite[188]{afmh}.
% This doesn't apply to data transfer between main memory and caches to the extent it does
% when accessing \glspl{hdd}.
While this is \alts{plausible, sensible, reasonable} when accessing \glspl{hdd}, it
doesn't apply to data transfer between main memory and caches to nearly the same extent.
%
Some of these shortcomings are addressed by refined machine models, which include more
details of real caches~\cite[178]{afmh}.
% I.e., CMM (cache memory model) and IMM (internal memory model).  There doesn't seem to
% be much literature about them [1].  Algorithms for Memory Hierarchies only cites a
% single paper that uses the IMM.  Don't discuss them.
%
% [1]: https://www.google.com/search?q=cache+memory+model+cmm
This complicates mathematical analysis~\cite[181]{afmh} and heuristics may be
used~\cite[191]{afmh}, which in turn \alts{exacerbates, amplifies} the need for
accompanying measurements~\cite[181]{afmh}.

\x{
\alts{
  {\alts[2]{On the contrary, Contrarily}, the \emph{\gls{com}} further increases the level
  of abstraction},
  {The opposite approach -- towards more abstraction -- is taken by the \gls{com}},
}.
}

% CMM and IMM add more detail ... bla bla bla ... COM actually abstracts further ...  bla
% bla ... counterintuitively this makes it more fitting in one aspect ... it works across
% the whole memory hierarchy ...

% vim: tw=90 sts=-1 sw=3 et fdm=marker

\subsection{Cache-Oblivious Model}
%           Model-Oblivious Algorithms
%           Ideal-Cache Model
%           \Gls{com}

% This one's kind of nice (and Scott Meyers seems to think it has merit).  It also has a
% Wikipedia page, unlike the EMM, CMM, or IMM (a very rough index of practicality).

% This seems like FUD:
%
% > Typical cache-efficient algorithms require tuning to several cache parameters which
% > are not always available from the manufacturer and often difficult to extract
% > automatically.
%      -- http://erikdemaine.org/papers/BRICS2002/paper.pdf

% > In computing, a cache-oblivious algorithm (or cache-transcendent algorithm) is an
% > algorithm designed to take advantage of a CPU cache without having the size of the
% > cache (or the length of the cache lines, etc.) as an explicit parameter.
%      -- https://en.wikipedia.org/wiki/Cache-oblivious_algorithm
%

The \emph{\gls{com}} or \emph{ideal-cache model}, introduced by
\textcite{coa-for-publication}, \alts{concedes, leaves, forfeits} most of the
aforementioned problems to empirical evaluation and further increases the level of
abstraction.
%
Algorithms for the \gls{com}, called \emph{cache-oblivious algorithms}, are designed
without the \alts{
  cache size \(M\) or block size \(B\),
  values of \(M\) and \(B\),
}
as parameters.  This seems silly since these values typically can be queried easily
\x{automatically} at both run and compile time~\cite[50]{drepper2007}
% See <https://stackoverflow.com/a/7284876>.  Also, cache line sizes are super homogeneous
% anyway.
but, perhaps counterintuitively, models one aspect of memory hierarchies better than the
\gls{emm}.
% > This model was born out of the necessity to capture the hierarchical nature of memory
% > organization. [...] Although there have been other attempts to capture this
% > hierarchical information the cache oblivious model seems to be one of the most simple
% > and elegant ones.
%      -- Algorithms for Memory Hierarchies, page 194
%
% > The ideal cache oblivious model enables us to reason about a two level memory like the
% > external memory model but prove results about a multi-level memory model.
%      -- Algorithms for Memory Hierarchies, page 195
%
% > One consequence is that, if a cache-oblivious algorithm performs well between two
% > levels of the memory hierarchy (nominally called cache and disk), then it must
% > automatically work well between any two adjacent levels of the memory hierarchy.
%      -- http://erikdemaine.org/papers/BRICS2002/paper.pdf (erikcom)
%
% > Good algorithms for this model give us good algorithms for all values of B and M.
% > They are especially useful for multi-level caches.
%      -- https://ocw.mit.edu/courses/electrical-engineering-and-computer-science/6-851-advanced-data-structures-spring-2012/calendar-and-notes/MIT6_851S12_L7.pdf
%
% > From a theoretical standpoint, the cache-oblivious model is appealing because it is
% > very clean.  A cache-oblivious algorithm is simply a RAM algorithm; it is only the
% > analysis that differs.  The cache-oblivious model also works well for multilevel
% > memory hierarchies, unlike the external-memory model, which only captures a two-level
% > hierarchy.
%      -- https://pdfs.semanticscholar.org/2f5a/76ccdc71971b746fbe7ff54db86c65a73e91.pdf
%
% > Frigo et al. showed that for many problems, an optimal cache-oblivious algorithm will
% > also be optimal for a machine with more than two memory hierarchy levels.
%      -- https://en.wikipedia.org/wiki/Cache-oblivious_algorithm
%
% > [W]e prove that an optimal cache-oblivious algorithm designed for two levels of memory
% > is also optimal for multiple levels.  We also prove that any optimal cache-oblivious
% > algorithm is also optimal in the previously studied HMM and SUMH models.
%      -- https://pdfs.semanticscholar.org/19ed/9795adc7204d3c9745b3c9f71f8496d26bb6.pdf
%
% > We prove that an optimal cache-oblivious algorithm designed for two levels of memory
% > is also optimal for multiple levels and that the assumption of optimal replacement in
% > the ideal-cache model can be simulated efficiently by LRU replacement.
%      -- http://supertech.csail.mit.edu/papers/FrigoLePr99.pdf#page=1, Abstract
%
An algorithm that performs well in the \gls{com} performs well across the entire memory
hierarchy~\cites[194\psq]{afmh}[4]{erikcom}; the same argument \alts[2]{\alts{for,
showing} asymptotically optimal movement of data, for data movement being
\x{asymptotically} \emph{optimal}} applies between any two levels of memory%
~\cite[lemma 15, \pno~10]{coa-paper}.
% TODO.  Is asymptotic analysis even useful, though?  The whole point of these models is
% that the hidden constants of "standard" asymptotic analysis matter.

% What does "optimal" mean?
%
% > An optimal cache-oblivious algorithm is a cache-oblivious algorithm that uses the
% > cache optimally (in an asymptotic sense, ignoring constant factors).
%      -- https://en.wikipedia.org/wiki/Cache-oblivious_algorithm
%
% > We say that [an algorithm] is *cache optimal* if the number of cache misses meet the
% > asymptotic lower bound for I/Os in the EMM for that problem.
%      -- Algorithms for Memory Hierarchies, page 181
%
% > [T]hese algorithms use an optimal amount of work and move data optimally among
% > multiple levels of cache.
%      -- http://supertech.csail.mit.edu/papers/FrigoLePr99.pdf#page=1, Abstract
%
% > For simplicity in this paper, we use the term "optimal" as a synonym for
% > "asymptotically optimal", since all our analyses are asymptotic.
%      -- http://supertech.csail.mit.edu/papers/FrigoLePr99.pdf#page=2, footnote
Optimal means that the asymptotic~\cite[2]{coa-paper} number of cache misses incurred
\x{by a cache-oblivious algorithm} matches the problem's lower bound in the \gls{com}.

% > In contrast to the external-memory model, algorithms in the cache-oblivious model
% > cannot explicitly manage the cache (issue block-read and block-write requests).  This
% > loss of freedom is necessary because the block and cache sizes are unknown.
%      -- http://erikdemaine.org/papers/BRICS2002/paper.pdf (erikcom), page 5
Cache misses take the place of \acrshortpl{io} \x{used in the \gls{emm}} because
cache-oblivious algorithms don't \alts{manage the cache, read or write cache lines}
explicitly.  This wouldn't be possible since the algorithms know neither the cache nor
the cache line size~\cite[5]{erikcom}.
% > The ideal cache uses the optimal off-line strategy of replacing the cache line whose
% > next access is furthest in the future
%      -- http://supertech.csail.mit.edu/papers/FrigoLePr99.pdf#page=1
Instead, the \gls{com} uses the optimal replacement strategy of evicting the cache line
that won't be accessed for the longest time in the future (Bélády's Algorithm).  This
\alts{
  is strangely \alts{at odds, out of touch, unrealistic} with real-world caches that don't
  know the future,
  {seems like out-of-touch, theoretical ivory-tower nonsense},
}.
% > The ideal-cache model makes the perhaps-questionable assumptions that there are only
% > two levels in the memory hierarchy, that memory is managed automatically by an optimal
% > cache-replacement strategy, and that the cache is fully associative.  We address these
% > assumptions in Section 6, showing that to a certain extent, these assumptions entail
% > no loss of generality.
%      -- http://supertech.csail.mit.edu/papers/FrigoLePr99.pdf#page=2
%
% > LRU and FIFO replacement do just as well as optimal replacement up to a constant
% > factor of memory transfers and up a constant factor wastage of the cache.
%      -- http://erikdemaine.org/papers/BRICS2002/paper.pdf#page=6 (erikcom)
%
% > The same argument extends to a variety of other replacement strategies.
%      -- http://supertech.csail.mit.edu/papers/Prokop99.pdf#page=46
%
% > We show that algorithms with regular complexity bounds (Equation (7.1)) (including all
% > algorithms heretofore presented) can be ported to less-ideal caches incorporating
% > least-recently-used (LRU) or first-in, first-out (FIFO) replacement policies [24, p.
% > 378].
%      -- http://supertech.csail.mit.edu/papers/Prokop99.pdf#page=51
%
% > [I]n many cases [the COM] is provably within a constant factor of a more realistic
% > cache's performance.
%      -- https://en.wikipedia.org/wiki/Cache-oblivious_algorithm
%
% > [A]s long as the number of memory transfers depends polynomially on the cache size M,
% > then halving M will only affect the running time by a constant factor.
%      -- http://erikdemaine.org/papers/BRICS2002/paper.pdf#page=6 (erikcom)
%
% > Intuitively, algorithms that slow down by a constant factor when memory (M) is reduced
% > to half, are called regular.
%      -- Algorithms for Memory Hierarchies, page 196
However, \citeauthor{coa-thesis} proves that for many algorithms\footnote{%
   Those satisfying equation (7.1) in~\cite[46]{coa-thesis}.  If the number of cache
   misses incurred by the algorithm only \say{depends polynomially on the cache size
   \(M\)}, the \alts{equation, condition} is satisfied~\cite[6]{erikcom}.
   % Halving the cache size only increases the number of cache misses by a constant
   % factor?
}
it only increases cache misses by a constant factor compared to various feasible
replacement strategies%
% ~\cite[lemma 12, \pno~10]{coa-paper}.%
% ~\cite[corollary 13, \pno~10]{coa-paper}.%
~\cite[corollary 19, \pno~46]{coa-thesis}.%
% See <http://supertech.csail.mit.edu/papers/Prokop99.pdf#page=46>.
\footnote{%
   e.g. LRU, FIFO, and random replacement
}

% According to `coa-paper` (page 9), the "four major assumptions" are:
% *  automatic replacement
% *  optimal replacement
% *  two levels of memory
% *  full associativity
%
% All simplifications and assumptions (I think):
% *  inherited/adopted from the EMM
%    *  two levels of memory
%    *  full associativity
%    *  tall cache
%       *  "It is also commonly assumed in external-memory algorithms."
%             -- http://erikdemaine.org/papers/BRICS2002/paper.pdf#page=7
%       *  I don't grok this one.
% *  new
%    *  automatic, optimal replacement
%    *  inclusion property
%       *  "[T]he values stored in cache i are also stored in cache i+1"
%             -- http://supertech.csail.mit.edu/papers/FrigoLePr99.pdf#page=10
%
% TODO: reiterate all simplifications and assumptions?

% > We show that the assumptions of two hierarchical memory models in the literature, in
% > which memory movement is programmed explicitly, are actually no weaker than ours.
% > Specifically, we prove (with only minor assumptions) that optimal cache-oblivious
% > algorithms in the ideal-cache model are also optimal in the hierarchical memory model
% > (HMM) [1] and in the serial uniform memory hierarchy (SUMH) model [5, 42].
%      -- http://supertech.csail.mit.edu/papers/Prokop99.pdf#page=12
%
% Exactly the same text is in the paper.  But only in the earlier version [1].
% [1]: https://pdfs.semanticscholar.org/19ed/9795adc7204d3c9745b3c9f71f8496d26bb6.pdf
%
% > An optimal cache-oblivious algorithm whose cache-complexity bound satisfies the
% > regularity condition (14) can be implemented optimally in expectation in multilevel
% > models with explicit memory management.
%      -- http://supertech.csail.mit.edu/papers/FrigoLePr99.pdf#page=11, theorem 17
%
% > [W]e have shown that optimal cache-oblivious algorithms in the ideal-cache model are
% > also optimal in the hierarchical memory model (HMM).
%      -- http://supertech.csail.mit.edu/papers/Prokop99.pdf#page=56
%
\citeauthor{coa-thesis} further justifies the model by proving, \say{with only minor
assumptions}~\cite[12]{coa-thesis}, that cache-oblivious algorithms that are optimal in
the \gls{com} \alts{can by executed with an optimal amount of \acrshortpl{io}, are also
optimal} in the \gls{emm}~\cite[theorem 32, \pno~56]{coa-thesis}.  In other words, most
cache-oblivious algorithms can be systematically transformed into \alts{cache-aware,
\gls{emm}} algorithms that asymptotically require the same amount of memory transfers in
the \gls{emm} as the cache-oblivious variant in the \gls{com}.

% TODO: given our motivation that "constant factors matter", what have we been doing here?

% TODO.  A regular algorithm that isn't explicitly for a hierarchical memory model
% (doesn't explicitly move data between different levels of memory) is already
% cache-oblivious, right?

% Time to give an example algorithm.  Options:
% *  matrix transposition
% *  matrix multiplication
% *  searching a tree (Van Emde Boas layout vs. standard layout)
% *  sorting
%    *  not easy
%    *  good performance seems problematic
%    *  looks like the autors of [1] dit it, though (lazy funnelsort)
% *  something about linked lists?
%
% Maybe matrix transposition...
%
% [1] https://app.cs.amherst.edu/~ccmcgeoch/cs34/papers/a2_2-brodal.pdf
\subsubsection{Cache-Oblivious Matrix Transposition}

The straightforward way to transpose \alts{
   an \(m\times n\) matrix \(D\),
   a matrix \(D\in\mathbb{R}^{m\times{}n}\),
}
out-of-place is to use two loops like so:
\begin{minted}[autogobble]{c}
   for (int i = 0; i < m; ++i)
      for (int j = 0; j < n; ++j)
         E[j][i] = D[i][j];
\end{minted}
Assuming \(D\) and \(E\) are stored in row-major layout (as they would be in the C and
\cpp{} languages), the reads from \(D\) are sequential memory accesses but the writes to
\(E\) are not.

\alts{When \(D^\mathsf{T}\) has, For} sufficiently long rows (\(m > B\))\footnote{%
   \(D^\mathsf{T}\) will be an \(n\times m\) matrix, so \(m\) is its number of columns.
},
every consecutive access will be to a different cache line.  If it has sufficiently many
rows, no access will be to a cache line that is still loaded; once the inner loop
completes, the cache line holding \alts[2]{\mintinline{c}{E[0][i]}, \(B_{0,i}\)} will have
been evicted.%
\footnote{%
   assuming LRU or FIFO replacement % FIXME: but we use optimal replacement!
}
Therefore, this algorithm incurs \(\Theta(mn)\) cache misses.

The algorithm is \alts{by definition, technically} already cache-oblivious since it
doesn't use \(M\) or \(B\), but it is not optimal\x{ in the \gls{com}}.  We can do better
with a divide-and-conquer approach.
% The code may only be 72 characters wide (75 when counting the indentation that will be
% stripped).  Note that the array arguments will decay to pointers.
\begin{minted}[autogobble, mathescape]{c}
   // Transpose the submatrix $(d_{ij})_{i\in I,\:j\in J}$.
   void transpose(int I[2], int J[2], int D[m][n], int E[n][m]) {
      int num_rows = 1 + I[1] - I[0];
      int num_cols = 1 + J[1] - J[0];
      if (num_cols == 1 && num_rows == 1) {
         E[J[0]][I[0]] = D[I[0]][J[0]];
      } else if (num_cols <= num_rows) {
         // Horizontally slice D into two submatrices and recurse.
         transpose((int[2]){I[0], I[0] + num_rows / 2 - 1}, J, D, E);
         transpose((int[2]){I[0] + num_rows / 2, I[1]}, J, D, E);
      } else { /* Vertically slice D analogously... */ }
   }
\end{minted}
% // Transpose the submatrix $(d_{ij}),\: i\in I, j\in J$.
% // Transpose the submatrix $(d_{ij})_{I_0<i<I_1, J_0<j<J_1}$
% // Transpose the submatrix $D\left[...\right]$

\begin{figure}
   \centering
   \begin{tikzpicture}
      \datavisualization
      [scientific axes=clean,
       x axis={logarithmic,
               label={Matrix Size (\si{\mebi\byte})}, length=0.8\textwidth, ticks={
                  major at={1, 2, 4, 8, 16, 32, 64, 128, 256},
               },
              },
       y axis={include value=0, grid=at ticks, label={Speedup}, length=6cm,
         % ticks={major also at={1}}
       },
       visualize as scatter,
       scatter={style={mark=*, mark options={scale=.65}}}]
         data [read from file=xpose/speedup.csv, separator=\space];
   \end{tikzpicture}
   \caption{TODO}
   \label{fig:TODO}
\end{figure}

% > A cache-oblivious algorithm is simply a RAM algorithm; it is only the analysis that
% > differs.
%      -- https://pdfs.semanticscholar.org/2f5a/76ccdc71971b746fbe7ff54db86c65a73e91.pdf
%
% Algorithms are cache-oblivious by default.

% Work by Erik Demaine:
%
% [1]: http://erikdemaine.org/papers/BRICS2002/paper.pdf
%      "Cache-Oblivious Algorithms and Data Structures - 2002"
% [2]: https://web.archive.org/web/20160320051801/http://courses.csail.mit.edu/6.897/spring03/scribe_notes/L15/lecture15.pdf
%      "6.897: Advanced Data Structures - Spring 2003"
% [3]: https://pdfs.semanticscholar.org/2f5a/76ccdc71971b746fbe7ff54db86c65a73e91.pdf
%      "6.897: Advanced Data Structures - Spring 2005"
% [4]: https://ocw.mit.edu/courses/electrical-engineering-and-computer-science/6-851-advanced-data-structures-spring-2012/calendar-and-notes/MIT6_851S12_L7.pdf
%      "6.851: Advanced Data Structures - Spring 2012"

% vim: tw=90 sts=-1 sw=3 et fdm=marker


% [1]: https://en.wikipedia.org/wiki/Analysis_of_algorithms

% vim: tw=90 sts=-1 sw=3 et fdm=marker
