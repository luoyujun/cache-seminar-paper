% Performance Analysis; Run Time Analysis; Timing Methodology; Profiling Pitfalls; Run
% Time Measuring; Benchmarking Methodology; Performance Monitoring
\section{Profiling Methodology}
\label{app:meth}

% Programs: `time`, bad; `\time -v`, better; oprofile(1); gprof(1); perf; Valgrind's
% cachegrind?; ...

\begin{comment}
Plenty of things can go wrong when \alts{profiling, measuring \alts{execution, run}
times}.  These are the ones I was aware of and tried to account for while doing
measurements for this \article.
\begin{itemize}
   \item \emph{CPU frequency scaling and boosting}.
      CPU frequencies are usually dynamic these days and automatically adjusted based on
      at least workload and temperature.  This should be disabled when measuring execution
      times.  I used the cpupower(1) program.
      % See <https://wiki.archlinux.org/index.php/CPU_frequency_scaling> and
      % <https://github.com/torvalds/linux/tree/master/tools/power/cpupower>.
      \begin{minted}[autogobble]{text}
         # cpupower frequency-set -g performance
         $ ./benchmark
         # cpupower frequency-set -g schedutil
      \end{minted}
      %stopzone
      % See [5].  Not a problem when measuring cycles.
   \item \emph{Interrupts and context switches.}  The process being timed has to share
      resources with other processes and the operating system.  A context switch will not
      only increase wall time, it will also increase cycles because of TLB flushes and
      cache evictions. % See <https://en.wikipedia.org/wiki/Context_switch>.

      Processor shielding, % See [5].
      taking the minimum of several measurements, and assigning a very high priority can
      \alts{help here, mitigate these problems}.
      % This is likely better than using nice(1).
      \begin{minted}[autogobble]{text}
         # chrt -f 99 \time -v ./program
      \end{minted}
   \item \emph{CPU jumping.}
      % See <https://youtu.be/vrfYLlR8X8k?t=33m34s>.
      CPU pinning (setting processor/thread affinity) may help. % See [2] and [5].
   \item \emph{Cache warmth.}
      % See <https://youtu.be/vrfYLlR8X8k?t=41m00s>.
      When comparing different solutions on the same data set, later ones may benefit from 
      data having been loaded into cache already.  Checking whether swapping the order of
      measurements changes the results is a good idea.
\end{itemize}
These effects may \alts{add noise to, pollute} the results, render them irreproducible, or
invalidate them completely.

General mitigation/alleviation strategy: take the minimum execution time of \alts{all, a
number of} runs; all noise is additive (TODO: not the one caused by the cache being hot).
% See [7].
\end{comment}

All programs were compiled and run on Linux using the \gls{gcc} with
\mintinline{text}{-O2} and \mintinline{text}{-march=native}.
% Additionally,
I disabled CPU frequency scaling with \mintinline{text}{cpupower(1)},
% See <https://wiki.archlinux.org/index.php/CPU_frequency_scaling>.
reduced the number of running processes,%
\footnote{%
   I stopped Firefox (which added visible noise) and the X server, for example.
}
% (e.g. leaving Firefox open in the background added lots of noise),
and assigned a high priority to the benchmark process with the \mintinline{text}{chrt(1)}
command.

The \emph{\href{https://github.com/meribold/cache-seminar-paper/blob/master/makefile}
{makefile}} used to compile and run the measurements as well as the full source code of
the programs is available \alts{here, on GitHub}:
\url{https://github.com/meribold/cache-seminar-paper}.

\subsection{Measuring CPU cycles}
\label{app:cycles}

% Frequency scaling is still a problem here.  For example, when the program is memory
% bound, it may take fewer cycles at a lower frequency while completing in the same wall
% time.  I.e., not everything is proportional to CPU frequency.

% > OProfile leverages the hardware performance counters of the CPU.
%      -- http://oprofile.sourceforge.net/about/

% > `ocount` is an OProfile tool that can be used to count native hardware events
% > occurring in [...] a specific application.
%      -- http://oprofile.sourceforge.net/doc/getting-started-with-ocount.html

% A list of AMD family 14h CPU performance counters is available at [1].  `ophelp(1)` can
% also be used.
% [1]: http://oprofile.sourceforge.net/docs/amd-family14h-events.php

I used the \mintinline{text}{ocount(1)} event counting tool added to the
\citetitle{oprofile}~\cite{oprofile} project in version 0.9.9.
% `perf(1)` seems to be a newer alternative to OProfile [8].  TODO: use it?
%
% > Newer CPU models support a hardware performance counter mode which uses hardware logic
% > to record events without any active code needed
%      -- https://en.wikipedia.org/wiki/OProfile
It uses \emph{hardware performance counters}%
\footnote{%
   \url{https://en.wikipedia.org/wiki/Hardware_performance_counter}
}
to provide low-overhead performance information without source code modifications.
\alts{%
   {Essentially, this is the procedure:},
   The \x{exact} process is more or less
}

% diff <(gcc -march=native -E -v - </dev/null 2>&1) <(gcc -E -v - </dev/null 2>&1)
% gcc -O2 -fno-prefetch-loop-arrays -Q --help=optimizers

\begin{minted}[autogobble]{text}
   $ gcc -march=native -O2 program.c
   # cpupower frequency-set -g performance
   # chrt -f 99 ocount -e CPU_CLK_UNHALTED ./a.out
\end{minted}
%stopzone

% ~\cite[78\psqq]{drepper2007}.

% [1]: https://en.wikipedia.org/wiki/Profiling_(computer_programming)
% [2]: https://en.wikipedia.org/wiki/Processor_affinity
% [3]: https://en.wikipedia.org/wiki/OProfile
% [4]: https://en.wikipedia.org/wiki/Perf_(Linux)
% [5]: https://github.com/JuliaCI/BenchmarkTools.jl/blob/master/doc/linuxtips.md
% [6]: http://stackoverflow.com/q/9006596 "On `/usr/bin/time` for benchmarks and `perf`"
% [7]: https://youtu.be/vrfYLlR8X8k "Andrei Alexandrescu - Writing Fast Code"
% [8]: http://oprofile-list.sf.narkive.com/rVbJlHdX/oprofile-vs-perf
% [9]: https://serverfault.com/q/161008
%      "What is the difference between renice and chrt commands in Linux?"

% vim: tw=90 sts=-1 sw=3 et fdm=marker
