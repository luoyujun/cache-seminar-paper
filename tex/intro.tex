\section{Introduction}
% What is this paper about?  Why learn about them?  How much performance is at stake?

% What are hardware caches?  Why caches?
A hardware cache is a \alts{comparatively, relatively} fast and small physical memory.  It
stores a subset of the data present in slower, larger memory that is expected to be used
again soon.  The purpose of this additional memory is to reduce the number of accesses to
the underlying slower storage.

% Hardware caches aren't going away.
There are fundamental reasons that having one single, \alts{uniform, homogeneous} type of
memory is not viable.  No signal can propagate faster than the speed of light.  Thus,
every storage technology can only reach a finite amount of data within a desired access
latency~\cite[2]{afmh}.

The most ubiquitous example for hardware caches \alts{is the hierarchy, are the various
levels (most commonly 2 or 3)} of CPU caches that are found on almost all present-day
CPUs.  They are designated L1 cache, L2 cache, and so on, with L1 being the fastest and
smallest level.  The underlying storage for CPU caches is the main memory.

There are more storage levels that \alts{comprise, constitute} the \emph{memory hierarchy}
of a computer along with CPU caches and main memory.  For example \glspl{hdd} and
\glspl{ssd}.
% Also: registers, internal buffers of HDDs and SSDs, (tapes), ...
% Focus on CPU caches.  Why?
However, swapping to \glspl{hdd} and \glspl{ssd} continues to become somewhat less common
as main memory sizes increase.  Even non-server systems can currently support 64 GiB of
main memory, eliminating the need for swapping to disk under many workloads.

I will focus on how to use CPU caches effectively and the \alts{enabled, resulting}
performance gains in this \article{}.

% TODO: what about TLB?

% vim: tw=90 sts=-1 sw=3 et fdm=marker
