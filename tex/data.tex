% TODO: put this in a separate file and include it in the preamble?
\pgfplotstableset{
   every head row/.style={before row=\toprule, after row=\midrule},
   every last row/.style={after row=\bottomrule},
   % Define a reusable custom style called 'array size' for a table column of array sizes
   % in KiB.  The syntax is the same as passing the key-value pairs directly to
   % `columns/<col name>/.style`.
   array size/.style={
      assign column name={Array Size (KiB)},
      numeric as string type,
      % Using `divide by=1024` or `preproc/expr={##1/1024}` performs floating-point
      % math and introduces errors.  This abomination hacked together by trial and
      % error doesn't.
      preproc cell content/.code={%
         \pgfkeysgetvalue{/pgfplots/table/@cell content}\a
         \newcount\b
         \b=\number\a
         \divide\b by 1024
         \pgfkeyssetvalue{/pgfplots/table/@cell content}{\the\b}
      },
   },
   % Define a column style called 'cycles'.
   cycles/.style={
      assign column name={Cycles / Iteration},
      string type,
      column type={S[round-mode=places, round-precision=2]},
      % column type=c, dec sep align,
      % fixed, fixed zerofill, precision=3,
   },
   % Define a complete table style.
   array size vs cycles/.style={
      columns={x, y, x, y},
      columns/x/.style={array size},
      columns/y/.style={cycles},
      % Split the table into two (super?) columns.  See page 42 and 43 of the
      % pgfplotstable manual.
      display columns/0/.style={
         select equal part entry of={0}{2},
         column type={S[table-format=3.0]},
      },
      display columns/1/.style={select equal part entry of={0}{2}},
      display columns/2/.style={
         select equal part entry of={1}{2},
         column type={S[table-format=6.0]},
      },
      display columns/3/.style={select equal part entry of={1}{2}},
   },
}

\clearpage
\section{Data}
\label{app:data}

% This section only contains figures and tables.  People recommend not using floats for
% this use-case [1][2].  I'm using the approach from [1] to manually spread the tables and
% figures on each page evenly.  This is a little ugly since it requires manually using
% `\clearpage`.
%
% [1]: https://tex.stackexchange.com/a/169092
% [2]: https://tex.stackexchange.com/a/205257

% The `center` environment us just used as a hack to get some vertical padding...  TODO: I
% should probably figure out a better way to get the same padding as floats and use
% something like this:
%    {
%       \centering
%       ...
%    }

\vfill

% \begin{table}[hp]
\begin{center}
   \pgfplotstabletypeset[
      % See <https://tex.stackexchange.com/q/131081/#comment311347_137554>.
      assign column name/.code=\pgfkeyssetvalue{/pgfplots/table/column name}{{{#1}}},
      % every head row/.style={%
         % output empty row,
         % before row={%
         %    \\\toprule
         %    Array Sizes (KiB) & \multicolumn{2}{c}{Cycles / Iteration} \\
         % },
      % },
      array size vs cycles,
   ]{access-times/access-times.csv}
   % \caption{Access Times for Random Reads}
   \captionof{table}{Access Times for Random Reads}
   \label{tab:access-times}
% \end{table}
\end{center}

\vfill

% \begin{table}[hp]
\begin{center}
   \pgfplotstabletypeset[
      assign column name/.code=\pgfkeyssetvalue{/pgfplots/table/column name}{{{#1}}},
      array size vs cycles,
   ]{seq-access-times/access-times.csv}
   % \caption{Access Times for Sequential Reads}
   \captionof{table}{Access Times for Sequential Reads}
   \label{tab:seq-access-times}
% \end{table}
\end{center}

\vfill
\clearpage
% Add space add the start of the page.  See <https://tex.stackexchange.com/a/21436>.
\vspace*{\fill}%

% \begin{figure}[hp]
\begin{center}
   \tikz \datavisualization[array size vs cycles plot]
      data [read from file=seq-access-times/cpu-bound/access-times.csv, separator=\space];
   % \caption{CPU-bound Sequential Read Access}
   \captionof{figure}{CPU-bound Sequential Read Access}
   \label{fig:seq-access-cpu-bound}
% \end{figure}
\end{center}

\vfill

% \FloatBarrier

% vim: tw=90 sts=-1 sw=3 et fdm=marker
