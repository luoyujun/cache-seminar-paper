\section{Amazing Data}
\label{app:data}

\begin{center}
   \pgfplotstabletypeset[
      % See <https://tex.stackexchange.com/q/131081/#comment311347_137554>.
      assign column name/.code=\pgfkeyssetvalue{/pgfplots/table/column name}{{{#1}}},
      every head row/.style={%
         before row=\toprule,
         % output empty row,
         % before row={%
         %    \\\toprule
         %    Array Sizes (KiB) & \multicolumn{2}{c}{Cycles / Iteration} \\
         % },
         after row=\midrule,
      },
      every last row/.style={%
         after row=\bottomrule,
      },
      columns={x, y, x, y},
      columns/x/.style={
         assign column name={Array Size (KiB)},
         numeric as string type,
         % Using `divide by=1024` or `preproc/expr={##1/1024}` performs floating-point
         % math and introduces errors.  This abomination hacked together by trial and
         % error doesn't.
         preproc cell content/.code={%
            \pgfkeysgetvalue{/pgfplots/table/@cell content}\a
            \newcount\b
            \b=\number\a
            \divide\b by 1024
            \pgfkeyssetvalue{/pgfplots/table/@cell content}{\the\b}
         },
      },
      columns/y/.style={
         assign column name={Cycles / Iteration},
         string type,
         column type={S[round-mode=places, round-precision=2]},
         % column type=c, dec sep align,
         % fixed, fixed zerofill, precision=3,
      },
      % Split the table into two (super?) columns.  See page 42 and 43 of the
      % pgfplotstable manual.
      display columns/0/.style={
         select equal part entry of={0}{2},
         column type={S[table-format=3.0]},
      },
      display columns/1/.style={select equal part entry of={0}{2}},
      display columns/2/.style={
         select equal part entry of={1}{2},
         column type={S[table-format=6.0]},
      },
      display columns/3/.style={select equal part entry of={1}{2}},
   ]{access-times/access-times.csv}
   \captionof{table}{Amazing Data}
   \label{tab:access-times}
\end{center}

% vim: tw=90 sts=-1 sw=3 et fdm=marker
