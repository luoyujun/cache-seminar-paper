\clearpage
\section{Data}
\label{app:data}

% This section only contains figures and tables.  People recommend not using floats for
% this use-case [1][2].  I'm using the approach from [1] to manually spread the tables and
% figures on each page evenly.  This is a little ugly since it requires manually using
% `\clearpage`.
%
% One problem I had with floats is that they don't spread out evenly on the section's
% initial page (the one with the "Data" heading).  They only do it on the following pages
% because those are special, dedicated float pages (those denoted by the `p` placement
% specifier).
%
% [1]: https://tex.stackexchange.com/a/169092
% [2]: https://tex.stackexchange.com/a/205257

% The `center` environment us just used as a hack to get some vertical padding...  TODO: I
% should probably figure out a better way to get the same padding as floats and use
% something like this:
%    {
%       \centering
%       ...
%    }

\vfill

\begin{center}
   \documentclass[border=1pt]{standalone}

% Stack for generating tables from CSV files.  Loading is done by [pgfplotstable].  It can
% also round, format and post-process data.  See [1].
% [siunitx] adds the `S` column type for advanced aligning in tables like around the
% decimal marker.  Getting this to work is a bit tricky; see [2].  There is some support
% for aligning around the decimal marker in [pgfplotstable] (see [3]), but I couldn't get
% it to work.  One can also right-align numbers relatively to each other but still center
% the column as a whole with [siunitx] (see [4]).
% [pgfplotstable]: https://ctan.org/pkg/pgfplotstable
% [siunitx]: https://ctan.org/pkg/siunitx
% [1]: https://tex.stackexchange.com/q/146716
% [2]: https://tex.stackexchange.com/q/131081
% [3]: https://tex.stackexchange.com/q/276485
% [4]: https://tex.stackexchange.com/q/9203
\usepackage{pgfplotstable}
% This gets loaded by the `pgfplotstable` sub-package as far as I know.
%
% \usepackage{pgfplots}
%
% It then suggests:
%
%    Package pgfplots Warning: running in backwards compatibility mode (unsuitable t
%    ick labels; missing features). Consider writing \pgfplotsset{compat=1.14} into
%    your preamble
%
% So I guess I'll go that.
\pgfplotsset{compat=1.14}
\usepackage[binary-units]{siunitx} % Load `\kibi` and `\mebi` etc. (`binary-units`).
% These packages are recommended by [pgfplotstable].  The booktabs package adds the
% `\toprule`, `\midrule`, and `\bottomrule` commands for better looking tables.  See [5].
\usepackage{booktabs, array, colortbl}
% [booktabs]: https://ctan.org/pkg/booktabs
% [5]: https://en.wikibooks.org/wiki/LaTeX/Tables#Professional_tables

\pgfplotstableset{
   every head row/.style={before row=\toprule, after row=\midrule},
   every last row/.style={after row=\bottomrule},
   % Define a reusable custom style called 'array size' for a table column of array sizes
   % in KiB.  The syntax is the same as passing the key-value pairs directly to
   % `columns/<col name>/.style`.
   array size/.style={
      assign column name={Array Size (KiB)},
      numeric as string type,
      % Using `divide by=1024` or `preproc/expr={##1/1024}` performs floating-point
      % math and introduces errors.  This abomination hacked together by trial and
      % error doesn't.
      preproc cell content/.code={%
         \pgfkeysgetvalue{/pgfplots/table/@cell content}\a
         \newcount\b
         \b=\number\a
         \divide\b by 1024
         \pgfkeyssetvalue{/pgfplots/table/@cell content}{\the\b}
      },
   },
   % Define a column style called 'cycles'.
   cycles/.style={
      assign column name={Cycles / Iteration},
      string type,
      column type={S[round-mode=places, round-precision=2]},
      % column type=c, dec sep align,
      % fixed, fixed zerofill, precision=3,
   },
   % Define a complete table style.
   array size vs cycles/.style={
      columns={x, y, x, y},
      columns/x/.style={array size},
      columns/y/.style={cycles},
      % Split the table into two (super?) columns.  See page 42 and 43 of the
      % pgfplotstable manual.
      display columns/0/.style={
         select equal part entry of={0}{2},
         column type={S[table-format=3.0]},
      },
      display columns/1/.style={select equal part entry of={0}{2}},
      display columns/2/.style={
         select equal part entry of={1}{2},
         column type={S[table-format=6.0]},
      },
      display columns/3/.style={select equal part entry of={1}{2}},
   },
}

% vim: ft=tex tw=90 sts=-1 sw=3 et fdm=marker


% FIXME: DRY.
\pgfkeys{%
   /pgf/number format/.cd,
   1000 sep={\,},
   min exponent for 1000 sep=4,
}

\begin{document}
\pgfplotstabletypeset[
   % See <https://tex.stackexchange.com/q/131081/#comment311347_137554>.
   assign column name/.code=\pgfkeyssetvalue{/pgfplots/table/column name}{{{#1}}},
   % every head row/.style={%
      % output empty row,
      % before row={%
      %    \\\toprule
      %    Array Sizes (KiB) & \multicolumn{2}{c}{Cycles / Iteration} \\
      % },
   % },
   array size vs cycles,
]{access-times/access-times.csv}
\end{document}

% vim: ft=tex tw=90 sts=-1 sw=3 et fdm=marker

   \captionof{table}{Access Times for Random Reads}
   \label{tab:access-times}
\end{center}

\vfill

\begin{center}
   \pgfplotstabletypeset[
      assign column name/.code=\pgfkeyssetvalue{/pgfplots/table/column name}{{{#1}}},
      array size vs cycles,
   ]{seq-access-times/access-times.csv}
   \captionof{table}{Access Times for Sequential Reads}
   \label{tab:seq-access-times}
\end{center}

\vfill
\clearpage
% Add space at the start of the page.  See <https://tex.stackexchange.com/a/21436>.
\vspace*{\fill}%

\begin{center}
   \documentclass[tikz, border=1pt]{standalone}

\usetikzlibrary{datavisualization}
% \usetikzlibrary{datavisualization.formats.functions}

% Perform an integer division by 1024 on an argument in the format (scientific notation)
% provided to the `tick typesetting` key.  Amazing.  This took me half a day to write.
% See page 805 of the TikZ and PGF manual (version 3.0.1a).  We don't use
% `\pgfmathprintnumberto` because its result is in math mode, e.g. `$42$`.
% `pgfmathfloatparsenumber` allows arbitrary precision.
\def\kibtypesetter#1{%
   % \pgfmathprintnumberto[int trunc,1000 sep={}]{#1}{\a}
   \pgfmathfloatparsenumber{#1}%
   \pgfmathfloattoint{\pgfmathresult}%
   \pgfmathsetmacro{\a}{\pgfmathresult}%
   \newcount\b%
   \b=\number\a%
   \divide\b by 1024%
   \pgfmathprintnumber{\the\b}%
}

\def\emphkibtypesetter#1{%
   \ensuremath{\mathbf{\kibtypesetter{#1}}}%
}

% % This uses floating-point division.
% \def\kibtypesetter#1{%
%    \pgfmathparse{#1/1024}%
%    % \pgfmathdivide{#1}{1024}%
%    % \pgfmathdiv{#1}{1024}%
%    \pgfmathprintnumber{\pgfmathresult}%
% }

\tikzdatavisualizationset{
   array size vs cycles plot/.style={
      scientific axes=clean,
      x axis={
         logarithmic,
         ticks={
            /pgf/number format/int detect,
            major={
               tick typesetter/.code=\kibtypesetter{####1},
               at={
                  2048, 8192, 131072, 2097152, 8388608, 33554432, 134217728,
                  32768 as \emphkibtypesetter{32768},
                  524288 as \emphkibtypesetter{524288},
                  % 32768 as [style={font=\bfseries}],
                  % 32768 as \textbf{\kibtypesetter{32768}},
                  % 32768 as \ensuremath{\mathbf{\kibtypesetter{32768}}},
                  % 32768 as [tick typesetter/.code=\emphkibtypesetter],
               },
            },
            % minor={
               % style=black, tick text at low,
               % tick typesetter/.code={\kibtypesetter{##1}},
               % at={2048, 8192, 131072, 2097152, 8388608, 33554432, 134217728},
            % },
            minor at={4096, 16384, 65536, 262144, 1048576, 4194304, 16777216, 67108864},
         },
         grid={at={32768, 524288}},
         label={Array Size (KiB)},
         length=0.8\textwidth,
      },
      y axis={include value=0, label={Cycles / Iteration}, length=6cm, grid=at ticks},
      visualize as scatter,
      scatter={style={mark=*, mark options={scale=.65}}},
   },
}

% vim: ft=tex tw=90 sts=-1 sw=3 et fdm=marker


% FIXME: DRY.
\pgfkeys{%
   /pgf/number format/.cd,
   1000 sep={\,},
   min exponent for 1000 sep=4,
}

\begin{document}
\tikz \datavisualization[array size vs cycles plot]
   data [read from file=seq-access-times/cpu-bound/access-times.csv, separator=\space];
\end{document}

% vim: ft=tex tw=90 sts=-1 sw=3 et fdm=marker

   \captionof{figure}{CPU-bound Sequential Read Access}
   \label{fig:seq-access-cpu-bound}
\end{center}

\vfill

\FloatBarrier

% vim: tw=90 sts=-1 sw=3 et fdm=marker
