\section{Key Concepts}
%        Basic Principles
%        Key Terms (that sounds boring)

Some architectural properties of hardware caches lead to important concepts for using them
effectively.

\subsection{Cache Line} % or Cache Block
% https://en.wikipedia.org/wiki/CPU_cache#Cache_entries
%
% > On x86/x64, cache line is 64 bytes for many years now.
%      -- http://ithare.com/c-for-games-performance-allocations-and-data-locality/
% > In early caches these lines were 32 bytes long; nowadays the norm is 64 bytes.
%      -- Drepper, p. 15
% > It is not possible for a cache to hold partial cache lines.
%      -- Drepper, p. 16
% > A cache line is the "unit" of data you transfer to a cache.
%      -- http://www.cs.umd.edu/class/sum2003/cmsc311/Notes/Memory/introCache.html
\emph{Cache lines} or \emph{cache blocks} are the unit of data transfer between main
memory and cache.  They have a fixed size, which has been \say{64 bytes for many years} on
x86/x64 CPUs~\cites{ithare-paadl}[\href{https://youtu.be/WDIkqP4JbkE?t=21m41s}{21:41}]
{scott-meyers-talk}.%
\footnote{%
% > The original Pentium 4 processor also had an eight-way set associative L2 integrated
% > cache 256 KB in size, with 128-byte cache blocks.
%      -- https://en.wikipedia.org/wiki/CPU_cache#Example
% TODO: what about the block size of main memory?  Should it be the same?
Line sizes aren't \emph{necessarily} \alts{identical, homogenous} among a CPU's caches.
The Intel Pentium 4 processor had an \texttt{L1d} cache with \say{64 bytes per cache
line}~\cite[p.~9]{pentium4} but an \texttt{L2} cache with \say{128 bytes per cache
line}~\cite[p.~11]{pentium4}.}
\begin{comment}
   \multiplefootnoteseparator%
   % See <https://tex.stackexchange.com/a/71015>.
   \footnote{This can also be checked on the command line:
   \mintinline{bash}{cat /proc/cpuinfo | grep cache_alignment}}
\end{comment}
% > It means that as soon as you've accessed any single byte in a cache line, all the
% > other 63 bytes are already in L1 cache
%      -- http://ithare.com/c-for-games-performance-allocations-and-data-locality/
\alts{%
   This means accessing a single uncached 32-bit integer entails loading another 60
   adjacent bytes.,
   {This means when a single byte has to be loaded, another 63 adjacent bytes will be as
   well.},
   {When, for example, accessing a single byte that isn't already cached, another 63
   adjacent bytes will be loaded.}
}
% Even when compiling C++ for 64-bit systems, `int` is typically 32-bit.

My E-450 CPU is no exception and both of its data caches have 64-byte cache lines.%
\footnote{See \cref{app:cpuinfo}.}
% \begin{minted}[gobble=3]{bash}
%    $ getconf LEVEL1_DCACHE_LINESIZE; getconf LEVEL2_CACHE_LINESIZE
%    64
%    64
% \end{minted}
%stopzone
% \footnote{%
%    \mintinline{bash}!$ getconf LEVEL1_DCACHE_LINESIZE; getconf LEVEL2_CACHE_LINESIZE!\\
%    %stopzone
%    \noindent\mintinline{bash}!64!\\
%    \noindent\mintinline{bash}!64!
% }
We can verify this quite easily.  Consider \cref{lst:line-size}.  It loops over an array
with an increment given at compile time as \texttt{STEP} and measures the processor time.
\begin{center}
% \begin{listing}
   \inputminted[firstline=27]{c}{line-size/line-size.c}
   % \caption{%
   \captionof{listing}{%
      Loop over \mintinline{text}{array} with increment \mintinline{text}{STEP}}
   \label{lst:line-size}
% \end{listing}
\end{center}
The results for different values of \texttt{STEP} are plotted in \cref{fig:line-size}.
% Starting from a step size of 16, the time roughly halves every time the step size is
% doubled.  For the first 4 step sizes however, it is almost constant.
As expected, the time roughly halves whenever the step size is doubled --- but only from a
step size of 16.  For the first 4 step sizes, it is almost constant.

% The reason why the loops take the same amount of time has to do with memory.  The
% running time of these loops is dominated by the memory accesses to the array, not by the
% integer multiplications.
%    -- https://igoro.com/archive/gallery-of-processor-cache-effects/
This is because the run times are \alts{primarily due to, dominated by} memory accesses.
Up to a step size of 8, every 64-byte line has to be loaded.  At 16, the values we modify
are 128 bytes apart,%
\footnote{16 \texttt{int64\_t} values of 8 bytes each}
so every other cache line is skipped.  At 32, three out of four cache lines are skipped,
and so on~\cite[cf.][example 2]{gallery}.

% Talk about how the term cache line is commonly conflated to mean an appropriately sized
% and *aligned* block of main memory that can be loaded into a cache line.  This is how it
% works, right?  Or can we load blocks starting at arbitrary addresses into a cache line?
% One could even imagine loading memory that isn't even contiguous.  Most sources don't
% seem to clear any of this up.  See <https://stackoverflow.com/q/3928995> ("How do cache
% lines work?")
\begin{comment}
   The term cache line is also used to refer to \alts{identically, appropriately} sized
   blocks in main memory.
   % that may be loaded into cache.
   These blocks are fixed:
   % I.e., each byte in main memory falls exactly into one cache line.
   % The boundaries

   Cache lines in main memory are fixed.
   % Data is not loaded starting from arbitrary addresses, but only from addresses that
   % are multiples of the cache line size.
   Data blocks that are loaded don't start at arbitrary addresses, but at multiples of the
   cache line size.
\end{comment}
% See <https://en.wikipedia.org/wiki/Partition_of_a_set>.  XXX: I feel like there's some
% kind of conflation of ideas going on here.
Both cache and main memory can be thought of as being partitioned%
% \footnote{%
%    In the set-theoretic sense
% }
\ (in the set-theoretic\x{al}\ sense) % https://en.wiktionary.org/wiki/set-theoretic
into \alts[3]{cache line-sized blocks, blocks of that size, cache lines}.  \alts[2]{{That
is, d}, D}ata is \alts{not, neither} \alts{read or written, loaded} starting from
arbitrary main memory addresses,
% ...nor is it loaded into arbitrarily aligned blocks in cache.
but only from addresses that are multiples of the cache line size.
% The starting address will by a multiple of the cache line size.

% TODO.  Part of the memory address partially determines which cache line is used.  The
% binary logarithm of the number of cache sets is the amount of bits required to identify
% the cache set.  The specific line used from that set depends on the replacement policy.
% A number of least significant (lowest-order, right-most) bits are ignored, since all
% addresses that only differ in those go into the same cache line.  If the line size is
% 64, log_2(64) = 6 bits are ignored.  See Drepper, page 15.

% TODO: this probably has some implications about aligning data.

% TODO: maybe talk about the 64-bit bus width and burst mode as well.  And maybe about how
% a write requires a read if we only write part of a cache line and the optimization of
% adding dummy writes (I think Andrei talked about this).
%
% [1]: https://stackoverflow.com/q/39182060 "Why isn't there a data bus which is as wide
%      as the cache line size?"

% XXX: hacks!  Explicit placement specifier without `t` (top) to prevent the figure from
% interrupting the source code listing.
\begin{figure}[hbp]
   \centering
   \begin{tikzpicture}
      \datavisualization
      [scientific axes=clean,
       x axis={logarithmic,
               ticks={major={at={1 as \textbf{1}, 2 as \textbf{2}, 4 as \textbf{4},
                                 8 as \textbf{8}, 16, 64, 256, 1024}},
                      minor={at={32, 128, 512}}},
               label={Step size}, length=0.8\textwidth},
       y axis={logarithmic,
               % ticks={major={at={4, 16, 64, 256}}, minor={at={8, 32, 128}}},
               % ticks={major={at={6, 12, 24, 48, 92, 184}}},
               ticks={major={at={10, 20, 40, 80, 160, 320}}},
               label={Processor time (ms)}, length=6cm},
       visualize as scatter,
       scatter={style={mark=*, mark options={scale=.65}}}]
         % See <https://tex.stackexchange.com/q/198323>.
         data [read from file=line-size/line-size.csv, separator=\space];
   \end{tikzpicture}
   \caption{Processor times for running \cref{lst:line-size}}
   \label{fig:line-size}
\end{figure}

% vim: tw=90 sts=-1 sw=3 et fdm=marker

\subsection{Prefetching}
\label{sec:prefetch}

Consider a simplified version of \cref{lst:access-times} that, instead of using random
accesses, simply walks over the array sequentially.  It still follows the pointers to do
this, but the array is no longer shuffled.  The results of profiling this new program
\alts{as, in the same way as, just as, like} \cref{lst:access-times} before are
\alts{plotted, shown} in \cref{fig:seq-access-times}.%
\footnote{%
   \Vref{tab:seq-access-times} shows the numerical results.%
   % Numerical results are shown in \vref{tab:seq-access-times}.%
}

% XXX: consider possible effects of [software prefetching][1].  I think GCC doesn't enable
% this type of optimization unless `-fprefetch-loop-arrays` is explicitly specified; i.e.,
% none of the `-O` levels enables it.  See [2].  XXX: WRONG:
%
%    $ gcc -O2 -Q --help=optimizers | grep prefetch
%    -fprefetch-loop-arrays                [enabled]
%
% This is also interesting:
%
%    $ diff <(gcc -Q --help=optimizers) <(gcc -O2 -Q --help=optimizers)
%
% [1]: https://en.wikipedia.org/wiki/Cache_prefetching#Compiler_directed_prefetching
% [2]: https://gcc.gnu.org/onlinedocs/gcc/Optimize-Options.html#Optimize-Options

% XXX.  This section may be misleading: it could seem like it suggests all of the measured
% speedup is a result of prefetching.  It is a huge contributor, though.  The result from
% accessing all data in a cache line should at most be that of dividing the initial main
% memory access (about 200 cycles) between 8 separate reads.  Since reads from L1d still
% take about 3 cycles:
%    (200 + 7 * 3) / 8 = 221 / 8 = 27.625
% Since the actual numbers aren't much above 6 cycles, prefetching still has a huge effect
% (6.3 / 27.625 ~= 22.8 %).
\x{Compared to the nearly 200 cycles the random accesses caused, ...}
Until the working set size \alts{matches that of, exceeds} the \gls{l1d}, the access times
are virtually unchanged at 3 cycles, but exceeding the \gls{l1d} and hitting the \gls{l2}
\alts{increases this by, adds} no more than \alts{a single, one} cycle.
More \alts{strikingly, remarkably}, \alts{exceeding} the \gls{l2}
\alts{%
   has \alts{similarly limited, comparably little} effect,
   is \alts{com:comparably, similarly} inconsequential,
}.
The access time plateaus not much above 6 cycles \x{now} --- about \alts{\SI{3}{\percent},
3\%, 3 \%} of the maximum we saw for random reads.
 % This is in large part thanks to \emph{prefetching}.
Much of this can be explained by the improved use of cache lines: the penalty of loading a
cache line is distributed among 8 accesses now.  This \alts{%
   could at \alts{best, most} get us down to,
   can not get us down to less than,
}
\SI{12.5}{\percent}.
% \alts{In large part, To a high degree, To a great extend}, the improvements are due to
% \emph{prefetching}.
The missing improvements are due to \emph{prefetching}.

Prefetching is a \x{heuristic} technique by which CPUs \alts{predict \x{certain},
recognize predictable} access patterns and \alts{%
   preemptively push cache lines up the memory hierarchy before the program needs them,
   speculatively load data before the program needs it,
}.
% > This can work well only when the memory access is predictable, though.
%      -- Drepper, p. 23
% > Currently prefetch units do not recognize non-linear access patterns.
%      -- Drepper, p. 60
\alts{%
   {This can not work unless cache line access is predictable, though, which basically
   means \x{sequential} linear},
   {For this to work, cache line access has to be predictable, which usually means
   sequential},
   % This requires,
   % This only works if,
}%
~\cite[60]{drepper2007}.%
\footnote{%
   As an example, the most complicated \emph{stride pattern} my laptop's CPU can detect is
   % https://en.wikipedia.org/wiki/Hyphen#Suspended_hyphens
   one that skips over at most 3 cache lines (for- or backwards) and may alternate strides
   (e.g.  +1, +2, +1, +2, \ldots)~\cite[278]{14h}.
}

% > The purpose of prefetching is to hide the latency of a memory access.
% ...
% > Prefetching has one big weakness: it cannot cross page boundaries.
% ...
% > [R]egardless of how good the prefetcher is at predicting the pattern, the program will
% > experience cache misses at page boundaries
%      -- Drepper, p. 60

% > Prediction or explicit prefetching might also guess where future reads will come from
% > and make requests ahead of time; if done correctly the latency is bypassed altogether.
%      -- https://en.wikipedia.org/wiki/Cache_(computing)#Latency

% > [P]refetching [can] remove some of the costs of accessing main memory since it happens
% > asynchronously with respect to the execution of the program. It can [...] make the
% > cache appear bigger than it actually is.
%      -- Drepper, p. 14
Prefetching happens asynchronously to normal program execution~\cite[14]{drepper2007}
% > [T]he processor is able to hide most of the main memory and even L2 access latency by
% > prefetching cache lines into L2 and L1d.
%      -- Drepper, p. 23
and can therefore\x{, in principle,}\ almost completely hide the main memory latency%
~\cite[23]{drepper2007}.
This is not quite what we observe in \cref{fig:seq-access-times} because the CPU
\alts{performs, has to perform} little enough work for memory bandwidth to become the
bottleneck.
% XXX: the peak transfer rate of my ThinkPad's memory (DDR3-1333, I think) is much higher
% (should be 10666.67 MB/s) than the rate observed in this test (about 2 GB/s).
%    8 B / (6.2 / 1.65 GHz) = 8 * 1.65 GB/s / 6.2 = 105.6 ~= 2.13 GB/s
% What limits it?  How to achieve the theoretical maximum?  My understanding is that
% sequential read access like this should pretty much be the most efficient use of RAM
% possible.
%
% I compiled and ran the [STREAM][1] benchmark ([FAQ][2]) by [Dr. John D. McCalpin][3]
% recommended [here][4].  It gives similarly low data rates.
%
% [1]: https://www.cs.virginia.edu/stream/
% [2]: https://www.cs.virginia.edu/stream/ref.html
% [3]: http://www.cs.virginia.edu/~mccalpin/
% [4]: http://www.admin-magazine.com/HPC/Articles/Finding-Memory-Bottlenecks-with-Stream
% [5]: https://software.intel.com/en-us/articles/optimizing-memory-bandwidth-on-stream-triad
% [6]: https://www.nersc.gov/users/computational-systems/cori/nersc-8-procurement/trinity-nersc-8-rfp/nersc-8-trinity-benchmarks/stream/
% [7]: https://en.wikipedia.org/wiki/Memory_bandwidth
%
% TODO: I also tried [bandwidth](http://zsmith.co/bandwidth.html).
Adding some expensive operations like integer divisions every loop iteration changes that
and \alts{effectively, almost completely} levels the cycles spend per iteration across all
working set sizes.%
% I tested this.  The difference between L1d and L2 virtually disappears (~0.01 cycles)
% and exceeding the L2 increases the time per element by a single cycle.
\footnote{%
   % TODO.
   See \vref{fig:seq-access-cpu-bound}.
}

% Ubiquitous.

% > Hardware based prefetching is typically accomplished by having a dedicated hardware
% > mechanism in the processor that watches the stream of instructions or data being
% > requested by the executing program, recognizes the next few elements that the program
% > might need based on this stream and prefetches into the processor's cache.
%      -- https://en.wikipedia.org/wiki/Cache_prefetching#Types_of_cache_prefetching
\alts{What I described so far is, So far I described} \emph{hardware} prefetching.  It
uses dedicated silicon to automatically detect access patterns.  There is also
\emph{software} prefetching, which is triggered by special machine instructions that may
be inserted by the compiler or manually by the programmer.  Software prefetching is
discussed in~\cite{drepper2007}.

% > The idea of [the _mm_prefetch() intrinsic] function (actually an asm instruction from
% > x86/x64 instruction set) is to inform CPU that you're about to need certain memory
% > location.
%      -- http://ithare.com/c-for-games-performance-allocations-and-data-locality/2/

% https://gcc.gnu.org/onlinedocs/gcc/Optimize-Options.html#index-fprefetch-loop-arrays

% > Most of the time, prefetch just silently works behind the scenes, and I didn't see
% > cases when messing with prefetch at application-level would be worth the trouble.  At
% > least in theory, however, such cases do exist.
%      -- http://ithare.com/c-for-games-performance-allocations-and-data-locality/2/

% > The source for the prefetch operation is usually main memory.
%      -- https://en.wikipedia.org/wiki/Cache_prefetching

\begin{figure}
   \centering
   \documentclass[tikz, border=1pt]{standalone}

\usetikzlibrary{datavisualization}
% \usetikzlibrary{datavisualization.formats.functions}

% Perform an integer division by 1024 on an argument in the format (scientific notation)
% provided to the `tick typesetting` key.  Amazing.  This took me half a day to write.
% See page 805 of the TikZ and PGF manual (version 3.0.1a).  We don't use
% `\pgfmathprintnumberto` because its result is in math mode, e.g. `$42$`.
% `pgfmathfloatparsenumber` allows arbitrary precision.
\def\kibtypesetter#1{%
   % \pgfmathprintnumberto[int trunc,1000 sep={}]{#1}{\a}
   \pgfmathfloatparsenumber{#1}%
   \pgfmathfloattoint{\pgfmathresult}%
   \pgfmathsetmacro{\a}{\pgfmathresult}%
   \newcount\b%
   \b=\number\a%
   \divide\b by 1024%
   \pgfmathprintnumber{\the\b}%
}

\def\emphkibtypesetter#1{%
   \ensuremath{\mathbf{\kibtypesetter{#1}}}%
}

% % This uses floating-point division.
% \def\kibtypesetter#1{%
%    \pgfmathparse{#1/1024}%
%    % \pgfmathdivide{#1}{1024}%
%    % \pgfmathdiv{#1}{1024}%
%    \pgfmathprintnumber{\pgfmathresult}%
% }

\tikzdatavisualizationset{
   array size vs cycles plot/.style={
      scientific axes=clean,
      x axis={
         logarithmic,
         ticks={
            /pgf/number format/int detect,
            major={
               tick typesetter/.code=\kibtypesetter{####1},
               at={
                  2048, 8192, 131072, 2097152, 8388608, 33554432, 134217728,
                  32768 as \emphkibtypesetter{32768},
                  524288 as \emphkibtypesetter{524288},
                  % 32768 as [style={font=\bfseries}],
                  % 32768 as \textbf{\kibtypesetter{32768}},
                  % 32768 as \ensuremath{\mathbf{\kibtypesetter{32768}}},
                  % 32768 as [tick typesetter/.code=\emphkibtypesetter],
               },
            },
            % minor={
               % style=black, tick text at low,
               % tick typesetter/.code={\kibtypesetter{##1}},
               % at={2048, 8192, 131072, 2097152, 8388608, 33554432, 134217728},
            % },
            minor at={4096, 16384, 65536, 262144, 1048576, 4194304, 16777216, 67108864},
         },
         grid={at={32768, 524288}},
         label={Array Size (KiB)},
         length=0.8\textwidth,
      },
      y axis={include value=0, label={Cycles / Iteration}, length=6cm, grid=at ticks},
      visualize as scatter,
      scatter={style={mark=*, mark options={scale=.65}}},
   },
}

% vim: ft=tex tw=90 sts=-1 sw=3 et fdm=marker


% FIXME: DRY.
\pgfkeys{%
   /pgf/number format/.cd,
   1000 sep={\,},
   min exponent for 1000 sep=4,
}

\begin{document}
\tikz \datavisualization[array size vs cycles plot]
   data [read from file=seq-access-times/access-times.csv, separator=\space];
\end{document}

% vim: ft=tex tw=90 sts=-1 sw=3 et fdm=marker

   \caption{Access Times for Sequential Reads}
   \label{fig:seq-access-times}
\end{figure}

% \begin{figure}
%    \centering
%    \tikz \datavisualization[array size vs cycles plot]
%       data [read from file=seq-access-times/step8/access-times.csv, separator=\space];
%    \caption{TODO}
%    \label{fig:seq8-access-times}
% \end{figure}

% [1]: http://ithare.com/c-for-games-performance-allocations-and-data-locality/
% [2]: http://ithare.com/c-for-games-performance-allocations-and-data-locality/2/
% [3]: https://en.wikipedia.org/wiki/Cache_prefetching

% vim: tw=90 sts=-1 sw=3 et fdm=marker

\subsection{Locality of Reference}
%           Principle of Locality

% TODO: What's *data* locality?  Just another term for spatial locality?

% > [C]aches have proven themselves in many areas of computing because access patterns in
% > typical computer applications exhibit the locality of reference.
%      -- https://en.wikipedia.org/wiki/Cache_(computing)

% > Locality is [...] one type of predictable behavior that occurs in computer systems.
% > Systems that exhibit strong locality of reference are great candidates for performance
% > optimization through the use of techniques such as the caching, prefetching for memory
% > and advanced branch predictors at the pipelining stage of processor core.
%      -- https://en.wikipedia.org/wiki/Locality_of_reference

% > Realizing that locality exists is key to the concept of CPU caches as we use them
% > today. -- Drepper, p. 14

% > The memory hierarchy wouldn't be very effective if two facts about programs weren't
% > true.  Programs exhibit spatial locality and temporal locality.
%      -- http://www.cs.umd.edu/class/sum2003/cmsc311/Notes/Memory/introCache.html

% > Programs which have good spatial locality benefit from the fact that data are
% > transferred from main memory to cache in blocks, while programs which have good
% > temporal locality benefit from the fact that caches hold several blocks of data.
%      -- Algorithms for Memory Hierarchies, page 171

\alts{%
  {Two properties exhibited by \x{the memory access patterns of} computer code to varying
  degrees \alts{
    distinctly impact,
    are particularly \alts{essential, crucial} for,
    merit emphasis because of their impact on,
  } cache effectiveness:
  these are \emph{spatial locality} and \emph{temporal locality}.},
  {Spatial locality is a property \x{exhibited by} \alts{computer code exhibits, programs
  exhibit} to varying degrees.},
}
\alts{%
  {Both are measures of how well the code's memory access pattern matches certain
  principles.},
  {Both are measures of how well certain \alts{principles, ideals} about memory access are
  matched.\x{  For spatial locality, these are:}},
}

% This one goes first because it's really more fundamental than spatial locality.
\subsubsection{Temporal Locality}

% Temporal locality refers to the reuse of specific data, and/or resources, within a
% relatively small time duration.
%      -- https://en.wikipedia.org/wiki/Locality_of_reference

% > A sequence of references exhibits temporal locality of recently accessed data are
% > likely to be accessed again in the near future.
%      -- Algorithms for Memory Hierarchies, page 215

% > Even if the memory used over short time periods is not close together there is a high
% > chance that the same data will be reused before long (temporal locality).
%      -- Drepper, p. 14

% One access to a memory location \alts{suggests, deserves} another within a short time
% frame.
One access suggests another.  That is, \alts{once, previously} \alts{referenced, accessed}
memory locations tend to be used again within a short time frame.
%
This is \alts{really, effectively, in fact} the \alts{intrinsic, fundamental}
\alts[2]{raison d'être of memory hierarchies, motivation for having a memory hierarchy in
the first place}.  When \alts{a cache line, some data} is loaded \x{into cache} but not
accessed again before being evicted, the cache \alts{provided no benefit, was useless}.
% If every memory location were only used once

\subsubsection{Spatial Locality}

% > [S]patial locality refers to requests for data physically stored close to data that
% > has been already requested.
%      -- https://en.wikipedia.org/wiki/Cache_(computing)

% > Spatial locality refers to the use of data elements within relatively close storage
% > locations.  Sequential locality, a special case of spatial locality, occurs when data
% > elements are arranged and accessed linearly, such as, traversing the elements in a
% > one-dimensional array.
%      -- https://en.wikipedia.org/wiki/Locality_of_reference

% On spatial locality:
%
% > When a data item is accessed, it is likely that data items in sequential memory
% > locations will also be accessed.
%      -- https://en.wikibooks.org/wiki/Microprocessor_Design/Cache

% > Data accesses are also ideally limited to small regions.
%      -- Drepper, p. 14

% > Spatial locality: When a block is accessed, it should contain as much useful data as
% > possible.
%      -- Algorithms for Memory Hierarchies, page 9

% > A sequence of references exposes spatial locality if data located close together in
% > address space tend to be referenced close together in time.
%      -- Algorithms for Memory Hierarchies, page 215

% \alts{Spatial locality, It} is a measure of how
% \alts{%
%   % predictable the memory access \alts{is, patterns are}.
%   well two \alts{principles, assumptions} about memory access are matched:
% }
\begin{enumerate*}[font=\bfseries]
  \item For each accessed memory location, nearby locations are used as well within a
    short time frame.
  % \item Memory locations that are accessed are close to recently accessed ones.
  \item Memory is accessed sequentially.
\end{enumerate*}
% [S]patial locality is one of the principles on which caches are based.
%      -- Drepper, p. 15
We have already seen in the last two sections that caches \alts{%
  take advantage of \alts{both these principles, spatial locality} by design,
  are designed to take advantage of \alts{these principles, spatial locality},
% }~\cite[15]{drepper2007}:
}:
\begin{enumerate}[font=\bfseries]
  \item Data is loaded in blocks; subsequent accesses to locations in an already loaded
    cache line are basically free.
  \item \alts{Cache lines, Locations} from sequential access patterns are prefetched
    \alts{ahead of time, speculatively}.
\end{enumerate}

\subsubsection{Notes}
% \subsubsection{Instruction Cache}

% Talk about how instructions naturally have good spatial locality and are therefore less
% problematic and less of a target for optimizations from an application programmer's
% point of view.

% > Code always has quite good spatial and temporal locality.
%      -- Drepper, p. 31

% > [I]nstructions usually run sequentially (with the occasionaly branch or jump).  Since
% > instructions are contiguous in memory, they exhibit spatial locality.  When you run
% > one instruction, you're likely to run the next one too.  That's the nature of running
% > programs.
%      -- http://www.cs.umd.edu/class/sum2003/cmsc311/Notes/Memory/introCache.html

% On temporal locality:
%
% > For code this means, for instance, that in a loop a function call is made and that
% > function is located elsewhere in the address space.  The function may be distant in
% > memory, but calls to that function will be close in time.
%      -- Drepper, p. 14

% As said before, instruction cache \alts{tends to benefit less from, is less dependent
% on} optimization.

\alts{%
  Access to instructions,
  {\alts{Access to \x{the}, The} machine code itself, which is also cached, },
}
inherently has good spatial locality~\cite[31]{drepper2007} since \alts{they are,
instruction are, it is}
\alts{%
  \x{naturally} executed sequentially outside of jumps,
  mostly executed sequentially (with some jumps),
},
and good temporal locality\x{~\cite[31]{drepper2007}}\ because of loops and function
calls~\cite[14]{drepper2007}.
%
Programs with good locality are \alts{said to be, called} \emph{cache-friendly}.

% [1]: https://en.wikipedia.org/w/index.php?title=Cache_memory&oldid=778902517#Functional_principles_of_the_cache_memory

% Data is loaded from bigger, slower memory into smaller, faster memory (e.g. from main
% memory into the CPU cache) in \emph{blocks}.

% Moving down the memory hierarchy, access latencies increase faster than the
% \emph{obtainable} bandwidth: \say{[w]e can still achieve large bandwidths by accessing
% many close-by bits together [...].  Access to large \emph{blocks} [emphasis added] of
% memory is almost as fast as access to a single bit}.~\cite[2]{afmh}.

% Consider the program shown in \cref{lst:array-sum}.  It repeatedly loops over an array
% to compute the sum of its elements.  Before exiting, it prints the CPU time spent
% summing the array.

% vim: tw=90 sts=-1 sw=3 et fdm=marker

% TODO: Sections on:
% * Memory Access Patterns?
%    * \subsection{Memory Access Pattern}
\label{sec:map}

% vim: tw=90 sts=-1 sw=3 et fdm=marker

%    * This can probably be explained in the locality section.
% * Associativity?
%    * % https://en.wikipedia.org/wiki/CPU_cache#Associativity
\subsection{Associativity}
\label{sec:assoc}

% vim: tw=90 sts=-1 sw=3 et fdm=marker

% * Cache Miss Types?
%    * Associativity has to be explained before conflict misses.
%    * \subsection{Types of Cache Misses}

% Three types of cache misses according to [1] and [2].
% * Compulsory misses
% * Capacity misses
% * Conflict misses
% Drepper uses the same classification (pp. 34, 55), but doesn't really introduce it.
%
% [1]: https://en.wikibooks.org/wiki/Microprocessor_Design/Cache#Cache_Misses
% [2]: "Algorithms for Memory Hierarchies", Section 8.3.3, page 180
%
% Wikipedia is confusing, though:
% > There are three kinds of cache misses: instruction read miss, data read miss, and data
% > write miss.
%      -- https://en.wikipedia.org/wiki/CPU_cache#CACHE-MISS

% TODO: what happens when we miss?  How bad is it?  What about \gls{smt}?

% vim: tw=90 sts=-1 sw=3 et fdm=marker
 % TODO?
% * Structure of Cache Entries?
%    * Tag, data block, and flag bits ("dirty" bit and "valid" bit).
%    * Virtual (what's a logical address?) vs. physical index and tag.
%    * See <https://en.wikipedia.org/wiki/CPU_cache#Cache_entry_structure>.
% * Replacement Strategies?
% * Write Policies?
% * Exclusive versus Inclusive?
%    * https://en.wikipedia.org/wiki/CPU_cache#Exclusive_versus_inclusive

% vim: tw=90 sts=-1 sw=3 et fdm=marker
