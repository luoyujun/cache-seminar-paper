\section{Motivation} % So what?
% > Three things Really Matter for performance.  The first one is Algorithm, the second
% > one is your code being Non-Blocking, and the third one is Data Locality."
%      -- http://ithare.com/c-performance-common-wisdoms-and-common-wisdoms/

Hardware caches are managed by hardware directly.  They are generally opaque to the
operating system and other programs.  That is, software has no direct control over the
contents of a hardware cache.

% So what?  Why learn about hardware caches?  How much performance do I gain/lose
% depending on how cache-friendly my algorithm/code is?
\alts{Despite this, We will see that despite this}, two algorithms solving the same
problem with the same asymptotic complexity (in the same \(\Theta(g(n))\)) may differ in
performance by two orders of magnitude because of different \emph{memory access
patterns} (\cref{sec:map})~\cite{bigos}.  We will see an example of this in
\cref{sec:vvl}.

In a nutshell, hardware caches are ubiquitous but the performance improvements they
provide are conditional.
\begin{comment}
   To use them effectively,
   % To obtain optimal performance,
   algorithms must be designed and implemented with the architecture
   % design, structure, manner of functioning, inner workings, properties
   of hardware caches in mind.
\end{comment}
Effective use of hardware caches requires knowledge about \alts{how they work, their
architecture}.  Algorithms must be designed and implemented observing \alts{this
knowledge, their interactions with hardware caches}.

% How?

% vim: tw=90 sts=-1 sw=3 et fdm=marker
